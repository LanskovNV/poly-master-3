\documentclass[a4paper,12pt]{article}

% Этот шаблон документа разработан в 2014 году
% Данилом Фёдоровых (danil@fedorovykh.ru) 
% для использования в курсе 
% <<Документы и презентации в \LaTeX>>, записанном НИУ ВШЭ
% для Coursera.org: http://coursera.org/course/latex .
% Исходная версия шаблона --- 
% https://www.writelatex.com/coursera/latex/5.3

% В этом документе преамбула

%%% Работа с русским языком
\usepackage{cmap}					% поиск в PDF
\usepackage{hyperref}
\usepackage{mathtext} 				% русские буквы в формулах
\usepackage[T2A]{fontenc}			% кодировка
\usepackage[utf8]{inputenc}			% кодировка исходного текста
\usepackage[english, russian]{babel}	% локализация и переносы
\usepackage{indentfirst}
\frenchspacing

\DeclareSymbolFont{T2Aletters}{T2A}{cmr}{m}{it}

%\usepackage{algorithm}
%\usepackage{algpseudocode}
\usepackage[linesnumbered,ruled,vlined]{algorithm2e}
\RestyleAlgo{ruled}
\SetKwInput{KwData}{Input}
\SetKwInput{KwResult}{Output}

\usepackage{relsize}

\renewcommand{\epsilon}{\ensuremath{\varepsilon}}
\renewcommand{\phi}{\ensuremath{\varphi}}
\renewcommand{\kappa}{\ensuremath{\varkappa}}
\renewcommand{\le}{\ensuremath{\leqslant}}
\renewcommand{\leq}{\ensuremath{\leqslant}}
\renewcommand{\ge}{\ensuremath{\geqslant}}
\renewcommand{\geq}{\ensuremath{\geqslant}}
\renewcommand{\emptyset}{\varnothing}

%%% Дополнительная работа с математикой
\usepackage{amsmath,amsfonts,amssymb,amsthm,mathtools} % AMS
\usepackage{icomma} % "Умная" запятая: $0,2$ --- число, $0, 2$ --- перечисление

%% Номера формул
%\mathtoolsset{showonlyrefs=true} % Показывать номера только у тех формул, на которые есть \eqref{} в тексте.
%\usepackage{leqno} % Нумереация формул слева

%% Свои команды
\DeclareMathOperator{\sgn}{\mathop{sgn}}

%% Перенос знаков в формулах (по Львовскому)
\newcommand*{\hm}[1]{#1\nobreak\discretionary{}
{\hbox{$\mathsurround=0pt #1$}}{}}

%%% Работа с картинками
\usepackage{graphicx}  % Для вставки рисунков
\graphicspath{{images/}{images2/}}  % папки с картинками
\setlength\fboxsep{3pt} % Отступ рамки \fbox{} от рисунка
\setlength\fboxrule{1pt} % Толщина линий рамки \fbox{}
\usepackage{wrapfig} % Обтекание рисунков текстом

%%% Работа с таблицами
\usepackage{array,tabularx,tabulary,booktabs} % Дополнительная работа с таблицами
\usepackage{longtable}  % Длинные таблицы
\usepackage{multirow} % Слияние строк в таблице

%%% Теоремы
\theoremstyle{plain} % Это стиль по умолчанию, его можно не переопределять.
 
\theoremstyle{definition} % "Определение"

\theoremstyle{remark} % "Примечание"
\newtheorem*{nonum}{Решение}

\newtheoremstyle{break}%
{}%
{}%
{\itshape}%
{}%
{\bfseries}%
{}%
{\newline}
{}%
\theoremstyle{break}
\newtheorem{definition}{Определение}
\newtheorem{problem}{Задача}
\newtheorem{theorem}{Теорема}[subsection]
\newtheorem{assumption}{Предположение}[subsubsection]

%%% Программирование
\usepackage{etoolbox} % логические операторы

%%% Страница
\usepackage{extsizes} % Возможность сделать 14-й шрифт
\usepackage{geometry} % Простой способ задавать поля
	\geometry{top=25mm}
	\geometry{bottom=35mm}
	\geometry{left=35mm}
	\geometry{right=20mm}
 %
%\usepackage{fancyhdr} % Колонтитулы
% 	\pagestyle{fancy}
 	%\renewcommand{\headrulewidth}{0pt}  % Толщина линейки, отчеркивающей верхний колонтитул
% 	\lfoot{Нижний левый}
% 	\rfoot{Нижний правый}
% 	\rhead{Верхний правый}
% 	\chead{Верхний в центре}
% 	\lhead{Верхний левый}
%	\cfoot{Нижний в центре} % По умолчанию здесь номер страницы

\usepackage{setspace} % Интерлиньяж
%\onehalfspacing % Интерлиньяж 1.5
%\doublespacing % Интерлиньяж 2
%\singlespacing % Интерлиньяж 1

\usepackage{lastpage} % Узнать, сколько всего страниц в документе.

\usepackage{soul} % Модификаторы начертания

\usepackage[usenames,dvipsnames,svgnames,table,rgb]{xcolor}
\hypersetup{				% Гиперссылки
    unicode=true,           % русские буквы в раздела PDF
    pdftitle={Заголовок},   % Заголовок
    pdfauthor={Автор},      % Автор
    pdfsubject={Тема},      % Тема
    pdfcreator={Создатель}, % Создатель
    pdfproducer={Производитель}, % Производитель
    pdfkeywords={keyword1} {key2} {key3}, % Ключевые слова
    colorlinks=true,       	% false: ссылки в рамках; true: цветные ссылки
    linkcolor=red,          % внутренние ссылки
    citecolor=black,        % на библиографию
    filecolor=magenta,      % на файлы
    urlcolor=cyan           % на URL
}

\usepackage{csquotes} % Еще инструменты для ссылок

%\usepackage[style=authoryear,maxcitenames=2,backend=biber,sorting=nty]{biblatex}

\usepackage{multicol} % Несколько колонок

\usepackage{tikz} % Работа с графикой
\usepackage{pgfplots}
\usepackage{pgfplotstable}
\pgfplotsset{compat=1.17}


\author{Никита Лансков}
\title{Задача византийских генералов}
\date{\today}

\begin{document} % конец преамбулы, начало документа

\maketitle
\tableofcontents

\newpage

\section{Постановка задачи}

Пусть n «белых» генералов возглавляют армии в горах и готовятся атаковать
«чёрных» в долине. Для связи атакующие используют надёжный канал (например,
телефон), исключающий подмену сказанного. Однако из n генералов m являются
предателями и активно пытаются воспрепятствовать согласию лояльных генералов.
Согласие состоит в том, чтобы все лояльные генералы узнали о численности всех лояльных
армий и пришли к одинаковым выводам (пусть и ложным) относительно состояния
предательских армий. (Последнее условие важно, если генералы на основании полученных
данных планируют выработать стратегию, и необходимо, чтобы все генералы выработали
одинаковую стратегию.)

По результатам обмена каждый из лояльных генералов должен получить вектор
целых чисел длины n, в котором i-й элемент либо равен истинной численности i-й армии
(если её генерал лоялен), либо содержит дезинформацию о численности i-й армии (если её
генерал не лоялен). При этом векторы, полученные всеми лояльными командирами,
должны быть полностью одинаковы.

Необходимо обеспечить протоколы общения между главнокомандующими и
реализовать алгоритм решения задачи византийских генералов.

\section{Реализация}

\subsection{Описание алгоритма}

Алгоритм решения задачи византийских генералов:

В нашем случае количество неверных генералов не изменяется со временем. Для
такой постановки в 1982 году Лесли Лампорт предложил рекурсивный алгоритм, который
задачу для случая m предателей генералов сводит к случаю m - 1 предателя.

\subsubsection{Алгоритм Лэмпорта:}

Алгоритм OM(0):
\begin{enumerate}
    \item Генерал посылает каждому лейтенанту свое значение
    \item Каждый лейтенант использует значение, которое получает от генерала
\end{enumerate}

Алгоритм ОМ(m) > 0:
\begin{enumerate}
    \item Генерал посылает каждому лейтенанту свое значение
    \item Для каждого i пусть $v_i$ будет значением, которе лейтенант получает
        от генерала. Лейтенант i действует как генерал в алгоритме OM(m - 1),
        чтобы послать значение $v_i$ каждому из $n-2$ других лейтенантов.
    \item Для каждого i и каждого $j \neq i$ пусть $v_i$ будет значением,
        которое лейтенант i получил от лейтенанта j на шаге 2 (с использованием
        алгоритма (m-1)). Лейтенант i использует большинство значений 
        $(v_1, v_2, ..., v_n)$
\end{enumerate}

\subsubsection{Пример n=4, m=1:}
Подробно покажем пример решения для n = 4, m = 1

\textbf{1-й шаг.} Каждый генерал посылает всем остальным сообщение, в котором указывает
численность своей армии. Лояльные генералы указывают истинное количество, а предатели
могут указывать различные числа в разных сообщениях. Генерал 1 указал число 1 (одна
тысяча воинов), генерал 2 указал число 2, генерал 3 (предатель) указал трём остальным
генералам соответственно x, y, z (истинное значение – 3), а генерал 4 указал 4.

\textbf{2-й шаг.} Каждый формирует свой вектор из имеющейся информации:
Вектор генерала №1: (1,2, x, 4);
Вектор генерала №2: (1,2, y, 4);
Вектор генерала №3: (1,2,3,4);
Вектор генерала №4: (1,2, z, 4).

\textbf{3-й шаг.} Каждый посылает свой вектор всем остальным (генерал 3 посылает опять
произвольные значения).
После этого у каждого генерала есть по четыре вектора:

\begin{table}[ht!]
    \begin{center}
        \begin{tabular}{cccc}
            g1 & g2 & g3 & g4 \\
            (1,2,x,4) & (1,2,x,4) & (1,2,x,4) & (1,2,x,4) \\
            (1,2,y,4) & (1,2,y,4) & (1,2,y,4) & (1,2,y,4) \\
            (a,b,c,d) & (e,f,g,h) & (1,2,3,4) & (i,j,k,l) \\
            (1,2,z,4) & (1,2,z,4) & (1,2,z,4) & (1,2,z,4) \\
        \end{tabular}
    \end{center}
\end{table}

\textbf{4-й шаг.} Каждый генерал определяет для себя размер каждой армии. Чтобы
определить размер i-й армии, каждый генерал берёт (n-m) чисел — размеры этой армии,
пришедшие от всех командиров, кроме командира i-й армии. Если какое-то значение
повторяется среди этих (n - m) чисел как минимум (n - m - 1) раз, то оно помещается в
результирующий вектор, иначе соответствующий элемент результирующего вектора
помечается неизвестным (или нулём и т. п.).

Все лояльные генералы получают один вектор (1,2, f(x, y, z),4), где f(x, y, z) есть
число, которое встречается как минимум два раза среди значений (x, y, z), или
«неизвестность», если все три числа (x, y, z) различны. Поскольку значения x, y, z и
функция f у всех лояльных генералов одни и те же, то согласие достигнуто.

\subsection{Подробности реализации}



\subsubsection{Канальный уровень:}

Нам необходимо, чтобы все сообщения между генералами (лейтенантами) точно
были доставлены. Также нам важен порядок доставки сообщений. Потому для реализации
канала связи между генералами будем использовать протокол Go-Back-N.
Протокол канального уровня GBN реализован в файле channel\_protocol.py
В начале программы каждая пара генералов создаёт линию связи. При отправке
сообщения по каналу, в дело вступает отдельный поток, в задачу которого входит
организация доставки сообщения (отправка, контроль получения, повторная отправка в
случае необходимости). Этот поток «зашит» в канал связи, за счёт чего процессотправитель не блокируется на этапе отправки, абстрагируется от реализации протокола и
точно уверен в том, что сообщение дойдёт получателя.
Вероятность потери сообщений равно 0.3. Однако протокол всё равно гарантирует,
что сообщения будет доставлено.

\subsubsection{Сетевой уровень:}

Алгоритм византийских генералов подразумевает связь каждый с каждым. Так как
основной интерес для нас представляет реализация именно протоколов взаимодействия, то
организуем сетевой уровень аналогично лабораторной 2 (протокол OSPF).
Сетевой протокол OSPF описан в файле network\_protocol.py
Топология роутеров представляется в виде графа. (файл topology.py). Данная
структура поддерживает операции добавления/удаления узлов, добавление/удаление связей
между узлами. Также в ней содержится реализация поиска кратчайших путей при помощи
алгоритма Дейкстры.
В файле network\_protocol.py находятся реализации узлов (Router) и выделенного
узла (DR – DesignatedRouter). Для подключения к сети роутер сначала устанавливает связь
с DR, посылает DR своих соседей, запрашивает у DR текущую топологию сети, после чего
приступает к обработке сообщений.
Теперь опишем действия DR при подключении нового узла:
Создать связь.
Получить соседей, после чего необходимо отправить информацию о них все
прочим роутерам (кроме отправителя)
Выдать по запросу топологию сети новому узлу

\section{Пример работы программы}

Рассмотрим работу алгоритма на примере четырёх генералов, где третий оказался
предателем (нумерация будет начинаться с нуля).

В качестве сообщения каждый генерал отправляет свой порядковый номер.
Предатель отправляет случайное значение в отрезке [0, n-1]
Сообщения, которые получили генералы:

3: [0, 1, 2, 0]

1: [0, 1, 2, 1]

0: [0, 1, 2, 3]

2: [0, 1, 2, 0]

Сформированные наборы:

0: [[0, 1, 2, 3], [0, 1, 2, 1], [0, 1, 2, 0], [1, 1, 2, 2]]

3: [[0, 1, 2, 3], [0, 1, 2, 1], [0, 1, 2, 0], [2, 2, 1, 2]]

1: [[0, 1, 2, 3], [0, 1, 2, 1], [0, 1, 2, 0], [2, 1, 1, 1]]

2: [[0, 1, 2, 3], [0, 1, 2, 1], [0, 1, 2, 0], [2, 2, 1, 2]]

Результаты:

0: [0, 1, 2, None]

1: [0, 1, 2, 1]

2: [0, 1, 2, None]

3: [0, 1, 2, None]

Как мы видим все лояльные генералы получили идентичную информацию о
значениях друг друга. Следовательно, согласие достигнуто. Как мы видим, лишь первый
генерал смог сделать выводы о значении предателя, так как для него предатель случайно
повторил значение 1

\section{Результаты}

Была реализована программа на языке Python для моделирования взаимодействия
между генералами (независимыми узлами) на сетевом и канальном уровне. Также
программа содержит решение задачи византийских генералов (моделирование наличия
злоумышленника в сети).

Показана работоспособность алгоритма на примере n = 4 генералов и m = 1
злоумышленников.

\end{document}




