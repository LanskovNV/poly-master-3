\documentclass[a4paper,12pt]{article}

% Этот шаблон документа разработан в 2014 году
% Данилом Фёдоровых (danil@fedorovykh.ru) 
% для использования в курсе 
% <<Документы и презентации в \LaTeX>>, записанном НИУ ВШЭ
% для Coursera.org: http://coursera.org/course/latex .
% Исходная версия шаблона --- 
% https://www.writelatex.com/coursera/latex/5.3

% В этом документе преамбула

%%% Работа с русским языком
\usepackage{cmap}					% поиск в PDF
\usepackage{hyperref}
\usepackage{mathtext} 				% русские буквы в формулах
\usepackage[T2A]{fontenc}			% кодировка
\usepackage[utf8]{inputenc}			% кодировка исходного текста
\usepackage[english, russian]{babel}	% локализация и переносы
\usepackage{indentfirst}
\frenchspacing

\DeclareSymbolFont{T2Aletters}{T2A}{cmr}{m}{it}

%\usepackage{algorithm}
%\usepackage{algpseudocode}
\usepackage[linesnumbered,ruled,vlined]{algorithm2e}
\RestyleAlgo{ruled}
\SetKwInput{KwData}{Input}
\SetKwInput{KwResult}{Output}

\usepackage{relsize}

\renewcommand{\epsilon}{\ensuremath{\varepsilon}}
\renewcommand{\phi}{\ensuremath{\varphi}}
\renewcommand{\kappa}{\ensuremath{\varkappa}}
\renewcommand{\le}{\ensuremath{\leqslant}}
\renewcommand{\leq}{\ensuremath{\leqslant}}
\renewcommand{\ge}{\ensuremath{\geqslant}}
\renewcommand{\geq}{\ensuremath{\geqslant}}
\renewcommand{\emptyset}{\varnothing}

%%% Дополнительная работа с математикой
\usepackage{amsmath,amsfonts,amssymb,amsthm,mathtools} % AMS
\usepackage{icomma} % "Умная" запятая: $0,2$ --- число, $0, 2$ --- перечисление

%% Номера формул
%\mathtoolsset{showonlyrefs=true} % Показывать номера только у тех формул, на которые есть \eqref{} в тексте.
%\usepackage{leqno} % Нумереация формул слева

%% Свои команды
\DeclareMathOperator{\sgn}{\mathop{sgn}}

%% Перенос знаков в формулах (по Львовскому)
\newcommand*{\hm}[1]{#1\nobreak\discretionary{}
{\hbox{$\mathsurround=0pt #1$}}{}}

%%% Работа с картинками
\usepackage{graphicx}  % Для вставки рисунков
\graphicspath{{images/}{images2/}}  % папки с картинками
\setlength\fboxsep{3pt} % Отступ рамки \fbox{} от рисунка
\setlength\fboxrule{1pt} % Толщина линий рамки \fbox{}
\usepackage{wrapfig} % Обтекание рисунков текстом

%%% Работа с таблицами
\usepackage{array,tabularx,tabulary,booktabs} % Дополнительная работа с таблицами
\usepackage{longtable}  % Длинные таблицы
\usepackage{multirow} % Слияние строк в таблице

%%% Теоремы
\theoremstyle{plain} % Это стиль по умолчанию, его можно не переопределять.
 
\theoremstyle{definition} % "Определение"

\theoremstyle{remark} % "Примечание"
\newtheorem*{nonum}{Решение}

\newtheoremstyle{break}%
{}%
{}%
{\itshape}%
{}%
{\bfseries}%
{}%
{\newline}
{}%
\theoremstyle{break}
\newtheorem{definition}{Определение}
\newtheorem{problem}{Задача}
\newtheorem{theorem}{Теорема}[subsection]
\newtheorem{assumption}{Предположение}[subsubsection]

%%% Программирование
\usepackage{etoolbox} % логические операторы

%%% Страница
\usepackage{extsizes} % Возможность сделать 14-й шрифт
\usepackage{geometry} % Простой способ задавать поля
	\geometry{top=25mm}
	\geometry{bottom=35mm}
	\geometry{left=35mm}
	\geometry{right=20mm}
 %
%\usepackage{fancyhdr} % Колонтитулы
% 	\pagestyle{fancy}
 	%\renewcommand{\headrulewidth}{0pt}  % Толщина линейки, отчеркивающей верхний колонтитул
% 	\lfoot{Нижний левый}
% 	\rfoot{Нижний правый}
% 	\rhead{Верхний правый}
% 	\chead{Верхний в центре}
% 	\lhead{Верхний левый}
%	\cfoot{Нижний в центре} % По умолчанию здесь номер страницы

\usepackage{setspace} % Интерлиньяж
%\onehalfspacing % Интерлиньяж 1.5
%\doublespacing % Интерлиньяж 2
%\singlespacing % Интерлиньяж 1

\usepackage{lastpage} % Узнать, сколько всего страниц в документе.

\usepackage{soul} % Модификаторы начертания

\usepackage[usenames,dvipsnames,svgnames,table,rgb]{xcolor}
\hypersetup{				% Гиперссылки
    unicode=true,           % русские буквы в раздела PDF
    pdftitle={Заголовок},   % Заголовок
    pdfauthor={Автор},      % Автор
    pdfsubject={Тема},      % Тема
    pdfcreator={Создатель}, % Создатель
    pdfproducer={Производитель}, % Производитель
    pdfkeywords={keyword1} {key2} {key3}, % Ключевые слова
    colorlinks=true,       	% false: ссылки в рамках; true: цветные ссылки
    linkcolor=red,          % внутренние ссылки
    citecolor=black,        % на библиографию
    filecolor=magenta,      % на файлы
    urlcolor=cyan           % на URL
}

\usepackage{csquotes} % Еще инструменты для ссылок

%\usepackage[style=authoryear,maxcitenames=2,backend=biber,sorting=nty]{biblatex}

\usepackage{multicol} % Несколько колонок

\usepackage{tikz} % Работа с графикой
\usepackage{pgfplots}
\usepackage{pgfplotstable}
\pgfplotsset{compat=1.17}


\author{Никита Лансков}
\title{Реализация протокола динамической маршрутизации
    Open Shortest Path First}
\date{\today}


\begin{document} % конец преамбулы, начало документа

\maketitle
\tableofcontents

\newpage

\section{Постановка задачи}

Требуется разработать систему из неограниченного количества взаимодействующих
друг с другом маршрутизаторов, которые организуются в сеть и обестпечивают 
передачу сообщений от каждого маршрутизатора к каждому по кратчайшему пути.

Необходимо рассмотреть:
\begin{enumerate}
    \item Три вида топологии сети: линейная, кольцо, звезда.
    \item Перестройку таблиц достижимости при стохастических разрывах связи.
\end{enumerate}

\section{Реализация}

Система реализована на языке Python. Топология связей роутеров представляется
в виде орграфа. Веса всех ребер графа равны единице. Также выделен отдельный 
роутер (DR - designated router), который находится вне топологии и обеспечивает
маршрутизацию сообщений об изменениях в топологии.

В данной работе мы абстрагируемся от типа реализации канала связи. Важно, чтобы
сообщения приходили в том же порядке, в каком и отправляются( будем 
использовать протокол связи Go-Back-N)

Для подключения нового роутера к сети:
\begin{enumerate}
    \item Роутер устанавливает связь с DR
    \item Роутер отправляет DR сообщение с информацией о соседях
    \item Роутер запрашивает у DR текущую топологию сети
\end{enumerate}

Designated Router связан со всеми узлами одновременно.
Когда DR получает сообщение о подключении или отключении узла, он обновляет 
свою топологию, после чего отправляет всем узлам сообщения об изменении
топологии.

Так как ключевая задача этой работы - протокол маршрутизации, то мы рассмотрим
только сообщения имеющие отношения к топологии.

Возможные типы сообщений:
\begin{itemize}
    \item Для DR:
        \begin{enumerate}
            \item NEIGHBOURS([neighbours]) - запрос на добавление в топологию
                новых соседей. Номер узла (чьи соседи) определяется номером
                отправителя.
            \item GET\_TOPOLOGY() - запрос на получение от DR текущей топологии
                сети
            \item OFF() - сообщение об отключении роутера
        \end{enumerate}
    \item Для Router
        \begin{enumerate}
            \item NEIGHBOURS(i, $neighbours_i$) - сообщение от DR о
                необходимости добавления новых соседей для узла i
            \item SET\_TOPOLOGY(topology) - сообщение от DR с информацией о
                текущей топологии
            \item OFF(i) - сообщение от DR о необходимости исключения узла i 
                из топологии.
            \item PRINT\_WAYS() - запрос о выводе на печать текущих кратчайших 
                путей до всех узлов. Это сообщение не влияет на топологию.
        \end{enumerate}
\end{itemize}

Так как топология представлена в виде орграфа то роутер при получении сообщения
о добавлении новогоо узла проверяет, является ли новый узел его соседом, если 
да - посылает DR сообщение о добавлении нового соседа (если такого соседства
еще нет в топологии).

Все роутеры, в том числе и DR запускаются в отдельных потоках выполнения.

\section{Примеры работы программы}

\subsection{Линейная топология}

Рассмотрим пример работы программы для линейной топологии с тремя узлами.

nodes: [0, 1, 2]

neighbours: [[1], [0, 2], [2]]


dr(0): (NEIGHBORS, {neighbors(0)})

dr(0): (GET\_TOPOLOGY)

dr(1): (NEIGHBORS, {neighbors(1)}))

r(0) :(SET\_TOPOLOGY)

dr(1): (GET\_TOPOLOGY)

r(0) : (NEIGHBORS: {1, neighbors(1)})

dr(2): (NEIGHBORS:{neighbors(2)}))

r(2) : (NEIGHBORS: {1, neighbors(1)}))

r(1) : (SET\_TOPOLOGY)

r(0) : (NEIGHBORS: {2, neighbors(2)})

r(1) : (NEIGHBORS: {2, neighbors(2)})

dr(2): (GET\_TOPOLOGY)

r(2) : (SET\_TOPOLOGY)


Все 3 узла подключились к сети. Посмотрим на полученные кратчайшие пути:


0: [[0], [0, 1], [0, 1, 2]]

1: [[1, 0], [1], [1, 2]]

2: [[2, 1, 0], [2, 1], [2]]


Предположим, что отключился нулевой узел:

dr(0): (OFF: None)

r(2) : (OFF: 0)

r(1) : (OFF: 0)

Тогда новые кратчайшие пути:

0: [[0], [], []]

1: [[], [1], [1, 2]]

2: [[], [2, 1], [2]]

Видим, что нулевой узел ни с кем не связан. Пусть нулевой узел снова восстановил
связь:

dr(0): (NEIGHBORS: {neighbors(0)})

dr(0): (GET\_TOPOLOGY)

r(1) : (NEIGHBORS: {0, neighbors(0)})

r(0) : (SET\_TOPOLOGY)

r(2) : (NEIGHBORS: {0, neighbors(0)})

Далее восстанавливается связь 1 → 0

dr(1): (NEIGHBORS: [0])

r(2) : (NEIGHBORS: {1, [0]})

r(0) : (NEIGHBORS: {1, [0]})

А кратчайшие пути вернулись в состояние до отключения:

0: [[0], [0, 1], [0, 1, 2]]

1: [[1, 0], [1], [1, 2]]

2: [[2, 1, 0], [2, 1], [2]]

Притом, заметим, что в начале, нулевой узел подключился самом первым. Других
узлов в сети ещё не существовали. Потому информация о соседях нулевого узла была
отправлена только DR. При повторном подключении, пришлось разослать её всем
роутерам.

Аналогично можно построить и другие топологии. Отличие будет только в
определении соседей для каждого узла:

\subsection{Топология кольцо}

nodes: [0, 1, 2]

neighbors: [[2, 1], [0, 2], [1, 0]]

Минимальные пути:

0: [[0], [0, 1], [0, 2]]

1: [[1, 0], [1], [1, 2]]

2: [[2, 0], [2, 1], [2]]

после отключения 2 узла:

0: [[0], [], [0, 2]]

1: [[], [1], []]

2: [[2, 0], [], [2]]

\subsection{Топология звезда с центром узле с индексом 1}

nodes: [0, 1, 2, 3]

neighbors: [[1], [0, 2, 3], [1], [1]]

Минимальные пути:

0: [[0], [0, 1], [0, 1, 2], [0, 1, 3]]

1: [[1, 0], [1], [1, 2], [1, 3]]

2: [[2, 1, 0], [2, 1], [2], [2, 1, 3]]

3: [[3, 1, 0], [3, 1], [3, 1, 2], [3]]

После отключения центрального узла

0: [[0], [], [], []]

1: [[], [1], [], []]

2: [[], [], [2], []]

3: [[], [], [], [3]]

После отслючения четвертого узла

0: [[0], [0, 1], [0, 1, 2], []]

1: [[1, 0], [1], [1, 2], []]

2: [[2, 1, 0], [2, 1], [2], []]

3: [[], [], [], [3]]

\section{Результаты}

Была реализована программа для моделирования протокола динамической 
маршрутизации OSPF для неограниченого количества взаимодействующих друг с 
другом маршрутизаторов и стохастическими разрывами соединения.

Данная программа была проверена на трех топологиях, из чего был сделан вывод 
о ее корректной работе на топологиях: линейная, кольцо, звезда.

\end{document}

