\documentclass[a4paper,12pt]{article}

\input{../preamble.tex}

\author{Никита Лансков}
\title{Реализация протокола динамической маршрутизации
    Open Shortest Path First}
\date{\today}


\begin{document} % конец преамбулы, начало документа

\maketitle
\tableofcontents

\newpage

\section{Постановка задачи}

Требуется разработать систему из неограниченного количества взаимодействующих
друг с другом маршрутизаторов, которые организуются в сеть и обестпечивают 
передачу сообщений от каждого маршрутизатора к каждому по кратчайшему пути.

Необходимо рассмотреть:
\begin{enumerate}
    \item Три вида топологии сети: линейная, кольцо, звезда.
    \item Перестройку таблиц достижимости при стохастических разрывах связи.
\end{enumerate}

\section{Реализация}

Система реализована на языке Python. Топология связей роутеров представляется
в виде орграфа. Веса всех ребер графа равны единице. Также выделен отдельный 
роутер (DR - designated router), который находится вне топологии и обеспечивает
маршрутизацию сообщений об изменениях в топологии.

В данной работе мы абстрагируемся от типа реализации канала связи. Важно, чтобы
сообщения приходили в том же порядке, в каком и отправляются( будем 
использовать протокол связи Go-Back-N)

Для подключения нового роутера к сети:
\begin{enumerate}
    \item Роутер устанавливает связь с DR
    \item Роутер отправляет DR сообщение с информацией о соседях
    \item Роутер запрашивает у DR текущую топологию сети
\end{enumerate}

Designated Router связан со всеми узлами одновременно.
Когда DR получает сообщение о подключении или отключении узла, он обновляет 
свою топологию, после чего отправляет всем узлам сообщения об изменении
топологии.

Так как ключевая задача этой работы - протокол маршрутизации, то мы рассмотрим
только сообщения имеющие отношения к топологии.

Возможные типы сообщений:
\begin{itemize}
    \item Для DR:
        \begin{enumerate}
            \item NEIGHBOURS([neighbours]) - запрос на добавление в топологию
                новых соседей. Номер узла (чьи соседи) определяется номером
                отправителя.
            \item GET\_TOPOLOGY() - запрос на получение от DR текущей топологии
                сети
            \item OFF() - сообщение об отключении роутера
        \end{enumerate}
    \item Для Router
        \begin{enumerate}
            \item NEIGHBOURS(i, $neighbours_i$) - сообщение от DR о
                необходимости добавления новых соседей для узла i
            \item SET\_TOPOLOGY(topology) - сообщение от DR с информацией о
                текущей топологии
            \item OFF(i) - сообщение от DR о необходимости исключения узла i 
                из топологии.
            \item PRINT\_WAYS() - запрос о выводе на печать текущих кратчайших 
                путей до всех узлов. Это сообщение не влияет на топологию.
        \end{enumerate}
\end{itemize}

Так как топология представлена в виде орграфа то роутер при получении сообщения
о добавлении новогоо узла проверяет, является ли новый узел его соседом, если 
да - посылает DR сообщение о добавлении нового соседа (если такого соседства
еще нет в топологии).

Все роутеры, в том числе и DR запускаются в отдельных потоках выполнения.

\section{Примеры работы программы}

\subsection{Линейная топология}

Рассмотрим пример работы программы для линейной топологии с тремя узлами.

nodes: [0, 1, 2]

neighbours: [[1], [0, 2], [2]]


dr(0): (NEIGHBORS, {neighbors(0)})

dr(0): (GET\_TOPOLOGY)

dr(1): (NEIGHBORS, {neighbors(1)}))

r(0) :(SET\_TOPOLOGY)

dr(1): (GET\_TOPOLOGY)

r(0) : (NEIGHBORS: {1, neighbors(1)})

dr(2): (NEIGHBORS:{neighbors(2)}))

r(2) : (NEIGHBORS: {1, neighbors(1)}))

r(1) : (SET\_TOPOLOGY)

r(0) : (NEIGHBORS: {2, neighbors(2)})

r(1) : (NEIGHBORS: {2, neighbors(2)})

dr(2): (GET\_TOPOLOGY)

r(2) : (SET\_TOPOLOGY)


Все 3 узла подключились к сети. Посмотрим на полученные кратчайшие пути:


0: [[0], [0, 1], [0, 1, 2]]

1: [[1, 0], [1], [1, 2]]

2: [[2, 1, 0], [2, 1], [2]]


Предположим, что отключился нулевой узел:

dr(0): (OFF: None)

r(2) : (OFF: 0)

r(1) : (OFF: 0)

Тогда новые кратчайшие пути:

0: [[0], [], []]

1: [[], [1], [1, 2]]

2: [[], [2, 1], [2]]

Видим, что нулевой узел ни с кем не связан. Пусть нулевой узел снова восстановил
связь:

dr(0): (NEIGHBORS: {neighbors(0)})

dr(0): (GET\_TOPOLOGY)

r(1) : (NEIGHBORS: {0, neighbors(0)})

r(0) : (SET\_TOPOLOGY)

r(2) : (NEIGHBORS: {0, neighbors(0)})

Далее восстанавливается связь 1 → 0

dr(1): (NEIGHBORS: [0])

r(2) : (NEIGHBORS: {1, [0]})

r(0) : (NEIGHBORS: {1, [0]})

А кратчайшие пути вернулись в состояние до отключения:

0: [[0], [0, 1], [0, 1, 2]]

1: [[1, 0], [1], [1, 2]]

2: [[2, 1, 0], [2, 1], [2]]

Притом, заметим, что в начале, нулевой узел подключился самом первым. Других
узлов в сети ещё не существовали. Потому информация о соседях нулевого узла была
отправлена только DR. При повторном подключении, пришлось разослать её всем
роутерам.

Аналогично можно построить и другие топологии. Отличие будет только в
определении соседей для каждого узла:

\subsection{Топология кольцо}

nodes: [0, 1, 2]

neighbors: [[2, 1], [0, 2], [1, 0]]

Минимальные пути:

0: [[0], [0, 1], [0, 2]]

1: [[1, 0], [1], [1, 2]]

2: [[2, 0], [2, 1], [2]]

после отключения 2 узла:

0: [[0], [], [0, 2]]

1: [[], [1], []]

2: [[2, 0], [], [2]]

\subsection{Топология звезда с центром узле с индексом 1}

nodes: [0, 1, 2, 3]

neighbors: [[1], [0, 2, 3], [1], [1]]

Минимальные пути:

0: [[0], [0, 1], [0, 1, 2], [0, 1, 3]]

1: [[1, 0], [1], [1, 2], [1, 3]]

2: [[2, 1, 0], [2, 1], [2], [2, 1, 3]]

3: [[3, 1, 0], [3, 1], [3, 1, 2], [3]]

После отключения центрального узла

0: [[0], [], [], []]

1: [[], [1], [], []]

2: [[], [], [2], []]

3: [[], [], [], [3]]

После отслючения четвертого узла

0: [[0], [0, 1], [0, 1, 2], []]

1: [[1, 0], [1], [1, 2], []]

2: [[2, 1, 0], [2, 1], [2], []]

3: [[], [], [], [3]]

\section{Результаты}

Была реализована программа для моделирования протокола динамической 
маршрутизации OSPF для неограниченого количества взаимодействующих друг с 
другом маршрутизаторов и стохастическими разрывами соединения.

Данная программа была проверена на трех топологиях, из чего был сделан вывод 
о ее корректной работе на топологиях: линейная, кольцо, звезда.

\end{document}

