\documentclass[a4paper,12pt]{article}

\input{../preamble.tex}

\author{Никита Лансков}
\title{Реализация протокола автоматического запроса повторной передачи
       Go-Back-N и Selective Repeat}
\date{\today}


\begin{document} % конец преамбулы, начало документа

\maketitle
\tableofcontents

\newpage

\section{Постановка задачи}

Требуется разработать систему из двух агентов, способных обмениваться данными
друг с другом.

Требования к системе:

\begin{enumerate}
    \item Должна моделироваться ненадежность канала связи: с заданной
        вероятностью пакеты должны теряться при передаче.
    \item Должна обеспечиваться доставка получателю всех отправленных данных, 
        посредством протоколов автоматического запроса повторной передачи
        \textit{Go-Back-N} и \textit{Selective Repeat}
\end{enumerate}

\section{Реализация}

Система реализована на языке программирования \textit{Python}. Система 
организована в виде двух потоков выполнения: поток отправителя и поток 
получателя. Взаимодействие между ними реализовано в виде очередей сообщений.

Программа разделена на следующие составляющие:
\begin{itemize}
    \item \textbf{Sender} - отправитель, формирует сообщения с данными.
    \item \textbf{Reciever} - получатель, получает сообщения и сообщает о факте
        доставки.
    \item \textbf{MsgQueue} - канал коммуникации, который хранит сообщения 
        между отправкой и получением, а также имитирует их потерю.
\end{itemize}

Каждый пакет содержит информацию о своем порядковом номере в окне, уникальный
номер блока, а также свой статус (доставлен, потерян).

Система принимает следующие параметры:
\begin{itemize}
    \item \textbf{protocol} - протокол связи (GBN/SRP)
    \item \textbf{window\_size} - величина скользящего окна в выбранном 
        протоколе.
    \item \textbf{timeout} - время в секундах, после которого пакет считается
        утерянным в случае отсутствия подтверждения его доставки.
    \item \textbf{loss\_probability} - вероятность потери сообщения при
        передаче $\left((0, 1]\right)$
\end{itemize}

\section{Оценка эффективности протоколов}

Оценку эффективности протоколов будем проводить по двум параметрам:
\begin{enumerate}
    \item коэффициент эффективности $k = \dfrac{кол-во~всех~пакетов}{кол-во 
        ~переданных~пакетов}$
    \item Время от начала до конца передачи в секундах - $t$ 
\end{enumerate}

Для оценки проведем серию экспериментов с различными значениями размера окна
и вероятности потери пакетов. Во всех тестах количество передаваемых пакетов 
равно 100, $timeout = 0.2 c.$

\subsection{Зависимость от вероятности потери пакета}

\begin{table}[ht!]
    \begin{center}
        \caption{Зависимость эффективности протоколов от вероятности
        потери пакета при $w=3$}\label{tab:1}
        \begin{tabular}{|c|c|c|c|c|}
        \hline
        w & \multicolumn{2}{|c|}{Go-Back-N} & \multicolumn{2}{|c|}{Selective repeat}\\
        \hline
          & t & k & t & k \\ 
        \hline
        0.0	 & 0.53  & 1.00    & 0.36  & 1.00   \\ \hline
        0.1	 & 3.62  & 0.77    & 1.35  & 0.84   \\ \hline
        0.2	 & 9.80  & 0.53    & 1.96  & 0.65   \\ \hline
        0.3	 & 10.95  & 0.50    & 4.82  & 0.57  \\ \hline
        0.5	 & 19.18  & 0.36    & 5.62  & 0.44   \\ \hline
        0.6	 & 23.76  & 0.31    & 10.94  & 0.35   \\ \hline
        0.7	 & 48.66  & 0.18    & 19.89  & 0.22   \\ \hline
        0.8	 & 71.67  & 0.13    & 24.34  & 0.17   \\ \hline
        0.9	 & 174.72  & 0.06    & 78.28  & 0.07  \\ \hline
        \end{tabular}
    \end{center}
\end{table}

\begin{figure}[ht!]
\begin{minipage}[h]{0.5\linewidth}
\center{\includegraphics[width=\linewidth]{4.png} \\ Зависимость коэффициента
    эффективности от вероятности потери пакета при $w=3$}
\end{minipage}
\hfill
\begin{minipage}[h]{0.5\linewidth}
\center{\includegraphics[width=\linewidth]{1.png} \\ Зависимость времени
    передачи от вероятности потери пакета при $w=3$}
\end{minipage}
\end{figure} 

\newpage

\subsection{Зависитость от размера окна}

\begin{table}[ht!]
    \begin{center}
        \caption{Зависимость эффективности протоколов от размера
        окна при $p=0.2$}\label{tab:2}
        \begin{tabular}{|c|c|c|c|c|}
        \hline
        w & \multicolumn{2}{|c|}{Go-Back-N} & \multicolumn{2}{|c|}{Selective repeat}\\
        \hline
          & t & k & t & k \\ 
        \hline
        2	 & 7.14  & 0.77    & 6.20  & 0.73   \\ \hline
        3	 & 6.08  & 0.65    & 1.97  & 0.74   \\ \hline
        4	 & 5.30  & 0.58    & 1.67  & 0.68   \\ \hline
        5	 & 4.63  & 0.55    & 2.23  & 0.65   \\ \hline
        6	 & 6.03  & 0.42    & 1.79  & 0.55   \\ \hline
        7	 & 3.50  & 0.51    & 1.16  & 0.64   \\ \hline
        8	 & 5.15  & 0.38    & 1.14  & 0.51   \\ \hline
        9	 & 5.35  & 0.35    & 1.33  & 0.48   \\ \hline
        10	 & 3.86  & 0.41    & 1.51  & 0.53   \\ \hline
        \end{tabular}
    \end{center}
\end{table}

\begin{figure}[ht!]
\begin{minipage}[h]{0.5\linewidth}
\center{\includegraphics[width=\linewidth]{3.png} \\ Зависимость коэффициента
    эффективности от размера окна при $p=0.2$}
\end{minipage}
\hfill
\begin{minipage}[h]{0.5\linewidth}
\center{\includegraphics[width=\linewidth]{2.png} \\ Зависимость времени
    передачи от размера окна при $p=0.2$}
\end{minipage}
\end{figure} 

\section{Результаты}

По рассмотренным выше зависимостям можно сделать следующие выводы:

\begin{itemize}
    \item При малых вероятностях потери пакета эффективность протоколов 
        практически не отличается. Далее протокол \textit{Go-Back-N} все
        значительнее проигрывает протоколу \textit{Selective repeat}
    \item Зависимость от размера окна менее явная. Можно заметить, что для
        протокола \textit{Selective repeat} эффективность улучшается с
        увеличением окна. Протокол \textit{Go-Back-N} ведет себя более
        хаотично, но общая тенденция аналогична.
\end{itemize}

\end{document}
