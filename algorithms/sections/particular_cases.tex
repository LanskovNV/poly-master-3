%!TEX root=../main.tex

\subsection{Частные случаи}

\begin{definition}
    Степень вершины графа - количество ребер, инцидентных этой вершине.
\end{definition}

\begin{definition}
    Регулярный граф степени k - это граф, все вершины которого имеют степень k.
\end{definition}

\subsubsection{Регулярный граф степени $< 3$}

Самый простой частный случай - если мы имеем дело с регулярным графом
степени меньше 3.
Если в таком графе и есть цикл - то это цикл, в который входят все вершины.
Чтобы это проверить, достаточно пройтись по всем ребрам, что можно сделать 
за полиномиальное время.

\subsubsection{Регулярный граф степени 3}

Если же регулярный граф имеет степень 3 - то все не так однозначно.
В общем случае - задача MAS остается NP трудной для таких графов, но для 
некоторых особых ситуаций мы можем найти приближение с точностью как минимум
$\dfrac{8}{9}$, или даже точное решение за полиномиальное 
время.~\cite{digraph_3}

Пусть дан регулярный граф степени 3: $G=(V, E)$, для которого мы хотим найти
максимальный ациклический подграф $S \subseteq E$. Будем проводить рассуждения
в рамках следующих предположений.

\begin{definition}
    Длина цикла - число ребер, входящих в цикл.
\end{definition}

\begin{definition}
    Положительная (отрицательная) степень вершины графa - это число всех исходящих 
    (входящих) ребер. Обозначения: $d^+\{v\}, (d^-\{v\})$ 
\end{definition}

\begin{assumption}\label{asump1}
    Все вершины в графе G имеют положительную и отрицательную степени не меньше
    1 и суммарную степень 3.
\end{assumption}

\begin{proof}
    Если в G содержится вершина, у которой положительная или отрицательная
    степень равна нулю, то мы можем сразу включить все смежные с ней ребра 
    в S, так как они будут содержаться в любом максимальном ациклическом
    подграфе.
\end{proof}

\begin{assumption}
    Граф G не содержит циклов длины 2 и 3.
\end{assumption}

\begin{proof}
    Если мы имеем дело с неориентированными циклами, то мы можем договориться
    для каждого такого цикла включать в S все ребра цикла, при этом
    сами циклы удалить из рассмотрения. Удаление цикла длины три не добавит
    новых циклов, так как мы работаем в рамках предположения~\ref{asump1}.
    В случае ориентированных циклов длины 2 и 3, нам достаточно не включать в S 
    какое-то одно ребро цикла.
\end{proof}

\begin{definition}
    $\alpha$-ребро - ребро $(i,j)$, такое, что 
    $$d^-\{i\} = 2, d^+\{i\}=1 \qquad d^-\{j\} = 1, d^+\{j\}=2$$
\end{definition}

Если в графе нет $\alpha$-ребер, то мы можем найти максимальный ациклический 
подграф за полиномиальное время следующим алгоритмом:

...

Если в графе есть $\alpha$-ребра, то делаем следующее.

...

Понятно, что последний приведенный алгоритм будет давать апроксимацию 
$\dfrac{8}{9}$ в регулярном графе степени 3.

В \cite{digraph_3} также показано, что если более аккуратно выбирать 
$\alpha$-ребра, то мы можем получить алгоритм с точностью $\dfrac{11}{12}$.

