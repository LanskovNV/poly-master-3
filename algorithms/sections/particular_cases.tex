%!TEX root=../main.tex

\section{Частные случаи}

\begin{definition}
    Степень вершины графа - количество ребер, инцидентных этой вершине.
\end{definition}

\begin{definition}
    Регулярный граф степени k - это граф, все вершины которого имеют степень k.
\end{definition}

\subsection{Регулярный граф степени $< 3$}

Самый простой частный случай - если мы имеем дело с регулярным графом
степени меньше 3.
Если в таком графе и есть цикл - то это цикл, в который входят все вершины.
Чтобы это проверить, достаточно пройтись по всем ребрам, что можно сделать 
за полиномиальное время.

\subsection{Регулярный граф степени 3}

Если же регулярный граф имеет степень 3 - то все не так однозначно.
В общем случае - задача MAS остается NP трудной для таких графов. В 
~\cite{digraph_3} приводится приближенный алгоритм для таких графов,
с точностью приближения $\dfrac{8}{9}$,

Пусть дан регулярный граф степени 3: $G=(V, E)$, для которого мы хотим найти
максимальный ациклический подграф $S \subseteq E$. Будем проводить рассуждения
в рамках следующих предположений.

\begin{definition}
    Длина цикла - число ребер, входящих в цикл.
\end{definition}

\begin{definition}
    Положительная (отрицательная) степень вершины графa - это число всех исходящих 
    (входящих) ребер. Обозначения: $d^+\{v\}, (d^-\{v\})$ 
\end{definition}

\begin{assumption}\label{asump1}
    Все вершины в графе G имеют положительную и отрицательную степени не меньше
    1 и суммарную степень 3.
\end{assumption}

\begin{proof}
    Если в G содержится вершина, у которой положительная или отрицательная
    степень равна нулю, то мы можем сразу включить все смежные с ней ребра 
    в S, так как они будут содержаться в любом максимальном ациклическом
    подграфе.
\end{proof}

\begin{assumption}\label{asump2}
    Граф G не содержит циклов длины 2 и 3.
\end{assumption}

\begin{proof}
    Если мы имеем дело с неориентированными циклами, то мы можем договориться
    для каждого такого цикла включать в S все ребра цикла, при этом
    сами циклы удалить из рассмотрения. Удаление цикла длины три не добавит
    новых циклов, так как мы работаем в рамках предположения~\ref{asump1}.
    В случае ориентированных циклов длины 2 и 3, нам достаточно не включать в S 
    какое-то одно ребро цикла.
\end{proof}

\begin{definition}
    $\alpha$-ребро - ребро $(i,j)$, такое, что 
    $$d^-\{i\} = 2, d^+\{i\}=1 \qquad d^-\{j\} = 1, d^+\{j\}=2$$
\end{definition}

По лемме 2.1 из~\cite{digraph_3}, если регулярный граф степени 3 не содержит
$\alpha$-ребер, то все циклы в нем не содержат общих ребер. Это утрверждение
легко доказывается от обратного. Если мы рассматриваем два цикла в регулярном 
графе степени три без $\alpha$-ребер, то возможны два случая: они пересекаются
по одному ребру или содержат несколько общих ребер. В случае пересечерния по 
одному ребру очевидно, что это ребро пересечения обязано быть $\alpha$-ребром.
Если же пересечение состоит из нескольких ребер, то среди этих ребер всегда 
найдется $\alpha$-ребро. Это легко проверить следующим алгоритмом. Предположим,
что у нас направление движения по общему пути для двух циклов - сверху вниз
(для определенности). Мы будем пробовать привести такой пример графа, в котором
этот самый общий путь не содержал бы $\alpha$-ребер. Зная, что все вершины 
у нас степени 3, и так как это часть цикла, попробуем добавить ребра к вершинам
которые находятся внутри пути так, чтобы не появилось $\alpha$-ребер. Будем 
идти сверху вниз. При таком подходе мы чередуем добавление ребер в разных 
направлениях, при этом мы никак не можем избежать появления аьфа-ребра среди
общих ребер циклов. 

Если в графе нет $\alpha$-ребер, то мы можем найти максимальный ациклический 
подграф за полиномиальное время следующим алгоритмом. Мы просто находим цикл
в графе, выбрасываем произвольное ребро из этого цикла, остальные добавляем в 
максимальный ациклический подграф. Также после этого мы стягиваем 
соответствующие ребра в оставшемся графе таким образом, чтобы~\ref{asump1}
 и~\ref{asump2} оставались истинными.

Если в графе есть $\alpha$-ребра, то делаем следующее.

\begin{enumerate}
    \item Находим $\alpha$-ребро $e$ в графе
    \item Удаляем $e$ Добавляем все $E(e)$ и вершины с нулевой 
        положительной/отрицательной степенью в решение.
    \item стягиваем вершины у которых $d^+\{v\}=1,~d^-\{v\}=1$
\end{enumerate}

Под стягиванием вершин в данном случае подразумевается следующее. если у 
вершины i одно входящее и одно исходящее ребро, то мы выбрасываем эту вершину 
из рассмотрения вместе с инцидентными ребрами, при этом соединяем начало 
входящего ребра с концом исходящего ребра новым ребром напрямую.

Понятно, что последний приведенный алгоритм будет давать апроксимацию 
$\dfrac{8}{9}$ в регулярном графе степени 3. Если мы нашли $\alpha$-ребро е для
которого |C(e)| = 9 то мы решаем эту компоненту точно, а если нет - то с 
точностю 8/9, так как для каждого альфа-ребра на удаление одного ребра мы
добавляем в наше решение как минимум 8 ребер.

В \cite{digraph_3} также показано, что если более аккуратно выбирать 
$\alpha$-ребра, то мы можем получить алгоритм с точностью $\dfrac{11}{12}$.

\subsection{Другие частные случаи}

В~\cite{planar} показано, что если регулярный граф степени 3 также является 
планарным, то MAS можно решить за полиномиальное время. Также показано, что
если планарный граф имеет максимальную степень вершин больше 3, то задача вновь
становится NP-полной.

