%!TEX root=../main.tex

\section{Полиномиальный приближенный алгоритм}

Теперь рассмотрим несколько алгоритмов, приведенных в~\cite{rnd_alg}

\begin{algorithm}
    \caption{Простейший приближенный алгоритм для MAS}\label{alg:3}
    \KwData{G(V, A) - ориентированный граф, некоторая нумерация $\pi$ вершин в нем}
    \KwResult{$A' \subseteq A$ - подмножество множества дуг, такое что 
        индуцированный граф $G'(V, A')$ является ациклическим}
   
        $A_1 = \left\{(i, j) \in A | \pi(i) < \pi(j)\right\}$ \\
        $A_2 = \left\{(i, j) \in A | \pi(i) > \pi(j)\right\}$ \\
        \If{$|A_1| > |A_2|$}{
            \Return $A_1$
        }
        \Return $A_2$
\end{algorithm}

В простейшем алгоритме мы разбиваем множество дуг на две части: Часть с прямыми
ребрами и часть с обратными ребрами. Соответственно, и $A_1$ и $A_2$ - являются
ацикличными по построению, и при этом $\max\{|A_1|, |A_2|\} \geq |A| / 2$

Теперь рассмотрим ряд менее тривиальных алгоритмов, а также построим оценки.

\begin{definition}\label{def:11}
    Будем считать, что у нашего невзвешенного графа веса всех ребер 
    одинаковы и равны 1. Тогда $\forall i \in V:$ \\
    \begin{align*}
        w_i^{in}(S) &= \sum\limits_{j\in S}w_{ji} - число~входящих~дуг~
        в~вершину~i\\
        w_i^{out}(S) &= \sum\limits_{j\in S}w_{ij} - число~исходящих~дуг~из~
        вершины~i\\
    \end{align*}
\end{definition}

\begin{definition}
    С учетом~(\ref{def:11}) будем называть \textbf{весом графа} $G(V, A)$ число
    дуг $|A|$.
\end{definition}

\begin{definition}
    Будем говорить, что перестановка $\pi$ индуцирует подграф $G'=(V, A')$, 
    и $A'=\{(i, j) \in A | \pi(i) < \pi(j)\}$
\end{definition}

В следующем алгоритме мы также на выходе получаем перестановку, которая индуци
рует ациклический подграф с не менее чем половиной ребер от их общего числа
в исходном графе, так как это свойство сохраняется на каждом шаге цикла.

\begin{algorithm}[hbt!]
    \caption{Базовый алгоритм}\label{alg:4}
    \KwData{G(V, A) - ориентированный граф}
    \KwResult{Перестановка $\pi$, которая индуцирует ациклический подграф.}
        
        $S=V, l=1, u=n$ \\

        \While{$u \geq l$} {
            Выбираем $i \in S$ \\
            $S = S - \{i\}$. \\
            \If{$w_i^{in}(S) \leq w_i^{out}(S)$}{
                $\pi(i) = l$ \\
                $ l=l + 1$
            } \Else{
                $\pi(i) = u$ \\
                $ u=u - 1$
            }
        }
        \Return~$\pi$
\end{algorithm}

Теперь, воспользовавшись базовым алгоритмом~\ref{alg:4}, построим вероятностный
алгоритм. Будем считать, что исходный граф не содержит циклов размера 2.

\begin{algorithm}[hbt!]
    \caption{Вероятностный алгоритм}\label{alg:5}
    \KwData{G(V, A) - ориентированный граф}
    \KwResult{Перестановка $\pi$, которая индуцирует ациклический подграф.}
        Разбиваем $V$ на две части $V_1, V_2$, добавляя каждую вершину в каждую
        из частей с вероятностью 0.5. \\
        \For{$r \in \{1, 2\}$}{
            $A_r = \{(i, j) | i, j \in V_r \}$ \\
            $\pi_r$ - перестановка вершин из $V_r$ \\
            Применяем алгорим~\ref{alg:4} к $(V_r, A_r)$,
            вершины выбираем в порядке возрастания индексов.
        }
        \If{$|A_{(\pi_1, \pi_2)}| > |A_{(\pi_2, \pi_1)}|$}{
            \Return $(\pi_1, \pi_2)$
        } \Else{
            \Return $(\pi_2, \pi_1)$
        }
\end{algorithm}

\newpage

\begin{theorem}
    Пусть APX - число дуг в решении, полученном при помощи 
    алгоритма~\ref{alg:5}. Тогда
    \[
        APX = \left( 0.5 + \Omega(\dfrac{1}{\sqrt{d_{max}}}) \right)|A|
    \]
\end{theorem}

\begin{proof}
    $i \in V_r$.

    Определим ряд значений:
    \begin{align*}
        D_i^{in} &= |(j, i) \in A: j > i| \\
        D_i^{out} &= |(i, j) \in A: j > i| \\
        d_i^{in} &= |(j, i): j\in V_r, j > i| \\
        d_i^{out} &= |(i, j): j\in V_r, j > i| \\
    \end{align*}

    $d_i^{in}$ - биномиально распределенная случайная величина с параметрами
    $(0.5, D_i^{in})$, так как для каждой дуги инцидентной с $i$ вероятность 
    того, что другой ее конец содержится также в $V_r$ равна $0.5$
    Аналогично, $d_i^{out}$ - биномиально распределенная случайная величина
    с параметрами $(0.5, D_i^{out})$.

    Не умаляя общности, предположим что $D_i^{in} \geq D_i^{out}$. 
    Для $a \geq 0$:

    \[P \equiv Pr(|d_i^{in} - d_i^{out}| \geq a) \geq 
    Pr(d_i^{in} - d_i^{out} \geq a)\]

    Согласно нашему предположению о том, что исходный граф не содержит
    циклов длины 2, $d_i^{in}$ и $d_i^{out}$ - независимые случайные величины.
    Тогда:

    \[
        P \geq Pr(X_1 - X_2 \geq a) \geq
        Pr(X_1 \geq 0.5D_i^{in} + a, X_2 \leq 0.5D_i^{in}) =
        0.5Pr(X_1 \geq 0.5D_i^{in} + a) 
    \]

    Где первое неравенство следует из предположения, что $D_i^{in} \geq 
    D_i^{out}$.
    Установим $a = \dfrac{1}{2}(D_i^{in})^{1/2}$ - стандартное отклонение для 
    $X_1$. Тогда получим:

    \[
        Pr(|d_i^{in} - d_i^{out}| \geq a) \geq \beta, \beta > 0
    \]

    Обозначим $D_i = D_i^{in} + D_i^{out}$, тогда
    \begin{align*}
        D_i^{in} &\geq D_i / 2 \\
        a = \Omega(\sqrt{D_i}) &= \Omega(D_i / \sqrt{D_i}) = \Omega(D_i / 
        \sqrt{d_{max}}) \\
    \end{align*}

    На каждом шаге цикла мы присваиваем вершину i на следующую верхнюю или 
    нижнюю позицию, в зависимости от знака выражения $d_i^{in} - d_i^{out}$.
    Суммарное число дуг, индуцированных перестановкой $\pi_r$:

    \[
        \sum\limits_{i \in V_r}\max\{d_i^{in}, d_i^{out}\} = 
        \sum\limits_{i \in V_r}\left(\dfrac{d_i^{in} + d_i^{out}}{2} + 
        \dfrac{1}{2} |d_i^{in} - d_i^{out}|\right) = 
        0.5|A_r| + 0.5\sum\limits_{i\in V_r}|d_i^{in} - d_i^{out}|
    \]

    Получаем, что число вершин, индуцированное $\pi_r$, равно:
    \[
        APX_r = 0.5|A_r| + \Omega\left(\sum\limits_i
        \dfrac{D_i}{\sqrt{d_{max}}}\right)
    \]

    Так как $\sum_iD_i = |A|$, получаем:

    \[
        APX_1 + APX_2 = 0.5(|A_1| + |A_2|) + \Omega(\dfrac{|A|}{\sqrt{d_{max}}})
    \]
\end{proof}

В \cite{rnd_alg} также показано что алгоритм \ref{alg:5} будет работать за 
$O(d_{max}^3 + |A|)$.

Также, в качестве конструктивного алгоритма в~\cite{rnd_alg} рассматривается 
применение локального поиска к множеству соседних перестановок $V_{\pi}$,
которое определяется следующим образом:
\begin{align*}
    \pi' \in V_{\pi} &\Leftrightarrow \forall j, k: j<k, \\
    \pi' &= (\pi_1, ..., \pi_{j-1}, \pi_k, \pi_{j+1}, ..., \pi_{k-1}, \pi_j,
    \pi_{k+1}, ...) \\
\end{align*}















