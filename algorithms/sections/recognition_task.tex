%!TEX root=../main.tex

\section{Основная часть}
\subsection{Формулировка задачи распознавания, доказательство ее NP-полноты}

\textbf{Формулировка задачи распознавания} \\

\begin{problem}[Задача MAS]
    Дан конечный ориентированный граф $D = (V, A)$ и константа 
    $B \in \mathbb{N}$. Существует ли подмножество $A' \in A$, такое, что
    подграф $D = (V, A')$ не содержит циклов и $|A'| >= B$. 
\end{problem}

\textbf{Доказательство NP-полноты} \\

Чтобы показать, что задача MAS является NP-полной, требуется:

\begin{enumerate}
    \setlength{\itemindent}{1em}
    \item Показать, что $MAS \in NP$ \\
    \item Свести к MAS другую известную задачу, чья NP-полнота уже установлена 
\end{enumerate}

\begin{definition}
    Задача распознавания $P$ принадлежит классу $NP$ при схеме
    кодирования $c$, если $L\left[P, c\right] \in NP$
\end{definition}
\begin{definition}
    Язык $L$ принадлежит классу NP, если существует НМТ М, распознающая
    L, и многочлен $p \in \mathbb{Z}[x]$, такие, что время работы М
    на любом входе $x \in \Sigma^*$ не превосходит $p(|x|)$
\end{definition}
Таким образом, чтобы доказать что задача MAS является NP полной, нам 
достаточно убедиться в существовании недетерминированной машины 
тьюринга, которая бы распознавала язык этой задачи.

Действительно, в качестве подсказки достаточно взять набор нулей 
и единиц длины $|V|$, где каждое значение соответствует конкретной 
вершине $v \in V$, и единицы стоят на местах вершин, которые входят 
в максимальный ациклический подграф, а на местах оставшихся вершин - 
нули.
Для этого предлагаю свести к задаче MAS задачу о независимом множестве.
\begin{definition}
    G - конечный граф. $W \in V(G)$ - независимое множество, если \\
    $\forall u,v \in W (uv \notin E(G))$
\end{definition}
\begin{problem}[О независимом множестве]
    Дан конечный неориентированный граф $G$ и число $B \in \mathbb{N}$.
    Есть ли в $G$ независимое множество размера не менее B.
\end{problem}

Преобразуем неориентированный граф из зачачи о независимом множестве $G$
к ориентированному $D$ следующим образом:

\begin{align*}
    V(D) &= V(G) \\ 
    A(D) &= \{\{uv, vu\} ~|~ \forall uv \in E(G)\}
\end{align*}

Таким образом, мы строим граф на тех же вершинах, и для каждого ребра
исходного графа добавляем две разнонаправленные дуги в наш новый 
ориентированный граф.

При таком построении, если мы найдем в графе $G$ независимое множество $W$, то
мы также нашли бы максимальный ациклический подграф в $D$. Это правда, так как
добавление любой из оставшихся вершин в подграф появился бы цикл, так как
в графе G добавленная вершина была бы связана с одной или несколькими вершинами
из независимого множества $W$.

