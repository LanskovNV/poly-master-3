%!TEX root=../main.tex

\section{Точный экспоненциальный алгоритм}

\subsection{Описание алгоритма}

Рассмотрим теперь точный экспоненциальный алгоритм, для этого вначале приведем 
несколько определений.

Покажем как построить алгоритм для двойственной к MAS задаче:

\begin{problem}[FAS]
    Дан конечный ориентированный граф $G=(V, A)$. Нужно найти такое 
    подмножество $F\subseteq A$, такое чтобы G - F Был бы ациклическим графом и
    при этом $|F|$ была минимальной из всех возможных.
\end{problem}

Понятно, что если мы знаем решение для задачи FAS, то мы легко найдем и сам
максимальный ациклический подграф в исходном графе, как $G\left[A - F\right]$

Также введем функцию, которая упорядочивает вершины в графе:

\begin{definition}
    $\pi : V \rightarrow {1, ..., |V|}$ - функция, которая задает некоторую нумерацию на 
    множестве вершин V. При этом, $\forall a = (i, j) \in A$ :
    \begin{align*}
        \pi(i) < \pi(j) &\Rightarrow a - прямая\\ 
        \pi(i) > \pi(j) &\Rightarrow a - обратная\\ 
    \end{align*}
\end{definition}

С учетом функции $\pi$ можно немного переформулировать задачу FAS следующим 
образом:

\begin{problem}[FAS]
    Дан ориентированный граф $G(V, A)$. Найти такую перестановку $\pi$, что \\
    $ 
        \mathlarger{\sum}\limits_{((u, v) \in A ~\&~ \pi(u) > \pi(v)) }1
    $
     - минимальна. Иными словами, требуется найти такую перестановку, в которой
     количество обратных дуг минимально.
\end{problem}

Под оптимальной перестановкой будем понимать такую перестановку, в которой
число обратных дуг минимально.

\begin{definition}
    $X(S)$ - число обратных дуг в оптимальной перестановке для индуцированного 
    графа $G[S]$, $S \subseteq V$.
\end{definition}

Также $X(S)$ можно представить в виде следующей рекурсивной формулы:

\begin{equation}\label{xs}
    X(S) = \min\limits_{u\in S}{\left\{ X(S - u) +
    \sum\limits_{((u, v) \in A ~\&~ v\in (S-u))} \right\} }
\end{equation}

Покажем корректность рекурсивной формулы~(\ref{xs}). 
Важно понимать, что при каждом увеличении размерности, к примеру при переходе
от $S-u$ к $S$, мы при каждом рассмотрении новой вершины $u$ присваиваем есть
новый, самый большой номер в нумерации вершин. Соответственно, все дуги,
которые приходят из вершин множества $S-u$ в вершину $u$ будут являться
обратными по определению. Ну и чтобы найти минимальное количество обратных дуг
мы ищем минимум суммы минимального числа обратных дуг для множества без 
рассматриваемой вершины и числа всех дуг, которые приходят в рассматриваемую
вершину из вершин $S-u$.

Перейдем теперь непосредственно к алгоритму~\cite{exact_alg}. Ключевой структурой данных, 
которую мы будем использовать, будет массив $Y$ размерности $2^n \times 2$.
В ходе работы алгоритма будем рассматривать подмножества множества вершин
$S \subseteq V$, и $\forall S$:

\begin{align*}
    Y[S, 1] &= X(S) \\
    Y[S, 2] &= \{v | v \in V: X(S)~is~minimized~in~eq.~(\ref{xs})\}
\end{align*}

\begin{algorithm}[hbt!]

\caption{FAS}\label{alg:two}
\KwData{A directed graph G}
\KwResult{Size of a minimum feedback set}

    Let Y be a $2^n \times 2$ array indexed from 0 to $2^n - 1$ \\
    Initialize $Y[S, 1] =\infty$, $Y[S, 2] = 0$ for all subsets
    $S \subseteq V ~and~ S \neq \emptyset$ \\
    Initialize $Y[\emptyset, 1] = Y[\emptyset, 2] = 0$ \\
    \For{$S \subseteq V enumerated~in~increasing~order~of~cardinality$}{
        \For{$u \in V - S$}{
            $P = Y[S, 1] + \sum_{((u, v)\in A~\&~ v\in (S-u))}1$ \\
            \If{$P = Y[S \cup \{u\}]$}{
                $Y[S\cup\{u\}, 2] = Y[S\cup\{u\}, 2] \cup \{u\}$
            }
            \If{$P < Y[S \cup \{u\}]$}{
                $Y[S\cup\{u\}, 1] = P$\\
                $Y[S\cup\{u\}, 2] = u$
            }
        }    
    }
    \Return Y[V, 1]
\end{algorithm}

Таким образом в результате работы алгоритма мы получим:

\begin{align*}
    Y[V, 1] &-число~обратных~ребер~в~оптимальной~перестановке~для~G \\ 
    \bigcup\limits_{S \subseteq V}Y[S, 2][1] &- множество~вершин,~которые~инцидентны 
    ~обратным~дугам \\
                                 &в~минимальном~по~размеру~множестве~F
\end{align*}

\subsection{Оценка сложности}

\begin{theorem}
    $G(V, E)$ - ориентированный граф, $|V| = n$, $|A| = m$.
    Тогда размер множества обратных дуг может быть найден за время $O^*(2^n)$
    и с использованием $O^*(2^n)$ памяти.
\end{theorem}

\proof
В алгоритме мы для каждого подмножества S отрабатываем некоторое количество 
вершин за время $O^*(n)$.













