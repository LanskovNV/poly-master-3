\documentclass[a4paper,12pt]{article}

\input{preamble}

\date{\today}
\author{Никита Лансков}
\title{Стохастические модели и анализ данных\\
Работа по восстановлению зависимости}


\begin{document} % конец преамбулы, начало документа

\maketitle
\tableofcontents

\newpage

\section{Постановка задачи}
Требуется выбрать массив данных с интервальной неопределенностью и восстановить
линейную зависимость.

Модель данных будем искать в классе линейных функций

\begin{equation}
    y = \beta_1 + \beta_2x
\end{equation}

При условии: $\beta_2 > 0$

% Вот тут будет график данных

\section{Параметры модели}

\subsection{Предобработка данных}
Выберем область, которую будем рассматривать, например: 
$t \in [2.54e^6, 3.58e^6]$

% График с выбранным диапазоном

Также возьмем для примера не все точки, а только некоторые с определенным 
шагом.

% График с точками, которые будем рассматривать

% Табличка с выбранными значениями

\subsection{Линейная модель МНК для точечных значений}

\subsection{Модель для интервального случая}

\newpage
\section{Коридор совместных зависимостей}
\newpage
\section{Прогноз за пределы интервала}
\newpage
\section{Граничные точки множества совместности}
\newpage
\section{Заключение}
\newpage

\listoffigures
\listoftables
\bibliographystyle{unsrt}
\bibliography{references}

\end{document} % конец документа
